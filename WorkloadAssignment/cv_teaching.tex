% Ana Paula curriculum vitae (2019)
\documentclass[10pt]{resume}

\usepackage[letterpaper,text={5.5in,9in},left=1.9in,centering]{geometry} % margins

\renewcommand\labelitemi{\tiny$\bullet$} %make bullets of itemize smaller

\author{\quad{}\quad{}\quad{}Ana Paula Centeno}
\address{
  Department of Computer Science\\
  Rutgers University\\
  110 Frelinghuysen Road\\
  Piscataway, NJ 08854-8019\\
  USA
}{
  anapaula@cs.rutgers.edu\\
  (732) 874-3035\\
}

\begin{document}
\maketitle{}

%\begin{category}{Objective}
%  \citem Non Tenure Track Assistant Professor Reappointment
%\end{category}

\begin{category}{Education}
  \citemnobullet{\bf Ph.D.}, Computer Science, Rutgers University, Piscataway, NJ, 2019\\
  Thesis: \emph{Optimizing Task Scheduling in Emergency Departments.}

  \citemnobullet{\bf M.S.}, Computer Science, Stony Brook University, Stony Brook,
  NY, 2003.\\
  Project: \emph{Supine and Prone colon image registration.}

  \citemnobullet{\bf M.S.}, Computer Science, Federal University of
  Rio Grande do Sul, Porto Alegre, RS, Brazil, 1999. \\
  Thesis: \emph{Penelope - A hierarchical scheduler for OR-parallel logic
    systems.} 
  
  \citemnobullet{\bf B.S.}, Computer Science, Catholic University of Pelotas,
  Pelotas, RS, Brazil, 1997.  \\
  Thesis: \emph{Implementation and evaluation of the heterogeneous
    operating system HetNOS at Catholic University of Pelotas.}

\end{category}

\begin{category}{Experience}

  \citemnobullet{\bf Teaching}, Rutgers University, Piscataway, NJ, Fall 2014 - present.
  \begin{itemize}
    \item CS-LLC Great Ideas in Computer Science Seminar, Spring 2019 - present.
    \item Introduction to Computer Science, Fall 2015 - present.
    \item Data Structures, Fall 2015 - present.
    \item Computer Architecture, Fall 2014, Spring 2015.
  \end{itemize}

  \citemnobullet {\bf Student Mentoring}
    \begin{itemize}
    \item {\em FIGS Faculty Mentor}: mentoring peer
      instructor student on developing course material for FIGS
      seminar, Summer 2019.
    \item {\em Computer Science Living and Learning Community Advisor}: the
      mentoring role to the CS-LLC includes one seminar per academic
      year and review sessions during exams, this year I 
      accompanied the students to the Grace Hoper Conference, Fall 2018 - present.
    \item {\em SAS Mentor}, Fall 2018 - present.
    \item {\em Member of the CS Academic Advising Team}: this role
      includes 4 hours of academic advising to students per week,
      Spring 2016 - present.
    \item {\em HackHers and HackRU Judge}, Spring 2016 - present.
    \end{itemize}

    \citemnobullet{\bf Women in Computer Science Initiative}
    \begin{itemize}
    \item Member of Rutgers Women in Computer Science Initiative partnership
      between DIMACS, the Department of Computer Science, and the
      Douglass Residential College aimed at increasing the percentage of
      undergraduate women majoring and minoring in Computer Science at
      Rutgers and elsewhere.
    \item Head of coordinating committee.
    \end{itemize}
    
    \citemnobullet {\bf Academic Administrative Activities}
    \begin{itemize}
    \item Responsible for the {\em modernization and stardardization} of the
      Introduction to Computer Science course, Summer 2019.
    \item {\em Coordinator of the Introduction to Computer Science course}:
      responsible for the (a) organization of class and recitation
      material, assignments and exams, (b) coordination of 30 Learning
      Assistants and 30 graders, and (c) exam coordination, Spring 2018 - present.
    \end{itemize}
  \citemnobullet {\bf Student Advising and Research}
    \begin{itemize}
      \item {\em Member of the CS Education Research group}: early
        stages of education research, Spring 2019.
      \item {\em Visualization}: guiding graduate student on
        developing a study to quantifies the aid of visualization
        tools to CS1 and CS2 students in understanding data structures
        concepts.
      \item {\em Dynamic Recitation}: advised undergraduate student
        during development of a student-focused, goal oriented,
        recitation management platform.
    \item {\em Independent Study}: advised undergraduate students. 
    \end{itemize}

  
  \citemnobullet{\bf Teaching Assistant}, Rutgers University, Piscataway NJ, January 2006 - Spring 2014.
  \begin{itemize}
    \item Discrete Structures II, Spring 2014.
    \item Internet Technology, Spring 2012.
    \item Distributed Systems, Fall 2011.
    \item Operating Systems, January 2006 - May 2010.
  \end{itemize}

  \citemnobullet{\bf Research Intern}, Intel, Hillsboro, Oregon, USA, Summer 2007.\\
  Devised several hierarchical power capping algorithms. I applied the same
  set of algorithms to perform inter and intra cluster power allocation. In addition
  the same set of algorithms were used to allocate power within a server machine.

  \citemnobullet{\bf Research Intern}, Packet General Networks Inc., Stony Brook, NY, USA,
  September 2002 - May 2004.\\
  I worked on the design and implementation of a secured file system
  for the Linux kernel.
  
  \citemnobullet{\bf Research Intern}, Viatronix Inc., Stony Brook, NY, USA, January 2001
  - August 2002.\\
  Devised and implemented an algorithm for image registration of the human 
  colon Prone and Supine views. The challenge lay on the fact that the colon
  shape changes as the patient moves for the next image and that there are
  only two points of attachments (beginning and end of the colon) we can
  rely on.

  \citemnobullet{\bf Research Assistant}, Federal University of Rio Grande do
  Sul, Porto Alegre, RS, Brazil, March 1997 - July 1999.\\
  Research on parallel systems scheduling, resulting in a distributed
  scheduler for PLoSys (parallel prolog system).

  \citemnobullet{\bf Research Assistant}, Computer Science Department, Catholic
  University of Pelotas, Pelotas, RS, Brazil, March 1995 - February 1997.

\end{category}

\begin{category}{Publications}

  \citem Ana Paula Centeno, R. Martin, and R. Sweeney. "REDSim: A
  Spatial Agent-Based Simulation For Studying Emergency Departments."
  In {\bf Proceedings of the Winter Simulation Conference}, 2013.

  \citem Wei Zheng, Ana P. Centeno, Frederic Chong, and Ricardo
  Bianchini. "LogStore: toward energy-proportional storage servers."
  In {\bf Proceedings of International Symposium on Low Power
    Eletronics and Design (ISLPED)}, 2012.

  \citem T. Heath, A. P. Centeno, P. George, L. Ramos,
  Y. Jaluria, and R. Bianchini. "Mercury and Freon: Temperature
  Emulation and Management in Server Systems". In {\bf Proceedings of
    the International Conference on Architectural Support for
    Programming Languages and Operating Systems (ASPLOS)}, October
  2006.

  \citem A. P. Centeno and C. Geyer. Penelope: A
  model of a hierarchical scheduler for PLoSys (Parallel
  Logic System). In {\bf Proceedings of Brazilian Symposium of computers
    architecture and high performance processing}, pages 39-48, B\'uzios,
  RJ, Brazil, October 1998. Rio de Janeiro: COPPE/UFRJ,
  1998.

  \citem A. P. Centeno. A model of a hierarchical scheduler for
  the PLoSys system. In {\bf Proceedings of III Academic Week of CPGCC}, 
  pages 65-68, Porto Alegre, RS, Brazil, August 1998. CPGCC of UFRGS, 
  1998.

  \citem A. P. Centeno and C. Geyer. Scheduling
    of parallel logic programming systems. Porto Alegre: CPGCC da
  UFRGS, 1997. (TI-686). 

  \citem A. P. Centeno and J. Barbosa. Implementation and
  evaluation of HetNOS heterogeneous operating system in UCPel. In
  {\bf Proceedings of V Congress of Scientific Initiation}, Pelotas, RS,
  Brazil, November 1996. Pelotas:FURG/UFPel/UCPel.
  
  \citem A. P. Centeno and J. Barbosa. Implementation and
  evaluation of a heterogeneous operating system in UCPel. In
  {\bf Proceedings of VII Seminar of Scientific Initiation}, Porto Alegre, RS,
  Brazil, September 1996. Porto Alegre: UFRGS/RS. 

  \citem A. P. Centeno and C. \'Avila. Studies to
  obtain photo-realistics images by computers. In {\bf Proceedings of IV 
    Congress of Scientific Initiation}, Rio Grande, RS, Brazil, November 1995.
  Rio Grande: FURG/UFPel/UCPel, 1995. 

\end{category}

%\begin{category}{References}
%  \citemnobullet{\bf Richard Martin}\\
%  Department of Computer Science\\
%  Rutgers University\\
%  110 Frelinghuysen Road\\
%  Piscataway, NJ 08854-8019\\
%  E-mail: rmartin@cs.rutgers.edu

%  \citemnobullet{\bf Ricardo Bianchini}\\
%  Department of Computer Science\\
%  Rutgers University\\
%  110 Frelinghuysen Road\\
%  Piscataway, NJ 08854-8019\\
%  E-mail: ricardob@cs.rutgers.edu
  
%  \citemnobullet{\bf Abhishek Bhattacharjee}\\
%  Department of Computer Science\\
%  Rutgers University\\
%  110 Frelinghuysen Road\\
%  Piscataway, NJ 08854-8019\\
%  E-mail: abhib@cs.rutgers.edu

%  \citemnobullet{\bf Erez Zadok}\\
%  Computer Science Department\\
%  2313-B Computer Science Building\\
%  Stony Brook University\\
%  Stony Brook, NY  11794-4400\\
%  Email: ezk@cs.sunysb.edu

%\end{category}

\end{document} 
