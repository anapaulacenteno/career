\documentclass[11pt]{article}
\usepackage[letterpaper,margin=1in]{geometry}
\usepackage{xcolor}
\usepackage{fancyhdr}
%\usepackage{tgschola} % or any other font package you like

\pagestyle{fancy}
\fancyhf{}
\fancyhead[C]{%
  \footnotesize\sffamily
  \yourname\quad
  \textcolor{blue}{\itshape\yourweb}\quad
  \textcolor{black}{\youremail}}

\newcommand{\soptitle}{Service Statement}
\newcommand{\yourname}{Ana Paula Centeno}
\newcommand{\youremail}{anapaula@cs.rutgers.edu}
\newcommand{\yourweb}{https://www.cs.rutgers.edu/\~{}anapaula}

\newcommand{\statement}[1]{\par\medskip
  \underline{\textcolor{blue}{\textbf{#1:}}}\space
}

\usepackage[
  colorlinks,
  breaklinks,
  pdftitle={\yourname - \soptitle},
  pdfauthor={\yourname},
  unicode
]{hyperref}

\begin{document}

\begin{center}\LARGE\soptitle\\
\large \yourname\ - January 2024
\end{center}

\hrule
\vspace{1pt}
\hrule height 1pt

\bigskip
\bigskip

At Rutgers I have become more than a professor, I became a role model
that instills a sense of belonging to students and a mentor that
fosters interest for computer science.

Research has shown that female students will leave the field if they
don't feel a sense of belonging.  At Rutgers our graduating student
body comprises of only 18\% of females. Therefore, it is of utmost
importance to create an environment in CS courses and student
gathering places that is welcoming to all students regardless of race
or gender.

I have been working to (a) change the CS culture at Rutgers to a
welcoming and supportive environment and, (b) increase the quality and
quantity of support that the introduction course sequence offers to our
students. Students achieve higher grades when they feel supported in
their academic career. To increase student's sense of support I have
created additional course content support (over 100 learning objective
videos) and created a tutoring program for the introduction sequence.

Following are the main points for this evaluation:

\bigskip

\begin{itemize}

  \item I have been one of the {\bf leaders in the CS/DIMACS Living
    Learning Community that has brought in \$490K from the Gates
    Foundation} through the Khoury Center for Inclusive Computing at
    Northeastern University.

  \item I have been instrumental in the complete redesign of the {\bf
    CS111 (Fall 2019)} and {\bf CS112 (Spring 2021)} curricula, as
    part of an effort that has been directly supported by the Rutgers
    Chancellor's Office and Northeastern Grant.
    
  \item As the Advancing Women in Computer Science {\bf coordinating
    committee leader} at Rutgers, I am responsible for coordinating
    curricular revisions for the introduction course sequece and
    co-curricular programming.

  \item CS111 and CS112 {\bf course coordinator} responsible for
    synchronized course content across lectures, the organization of
    50 Learning Assistants and 40 graders. Current enrollment is
    3600 students per Academic Year.

  \item I have been working with the CS department and SAS leadership
    to introduce a 3-course introduction to CS sequence to the major
    that will benefit students without computing background.
    
  \item Created a mentorship program for freshman, Assignment Guru,
    that gives students the opportunity to gain skills while creating
    assignments that demonstrate the positive social impact of
    computing. Over 30 mentees per year.
    
  \item Submitted NSF grants to support the CS education research that
    I have been conducting.
    
  \item Volunteered as an academic advisor until Fall 2023.

  \item High student ratings: considering the reviews for both courses
    my weighted mean for each question presented to the students is
    higher than the corresponding course's, department's and level's
    weighted mean.
  
\end{itemize}

\end{document}
