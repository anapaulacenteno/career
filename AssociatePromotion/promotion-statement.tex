\documentclass[11pt]{article}
\usepackage[letterpaper,margin=1in]{geometry}
\usepackage{xcolor}
\usepackage{fancyhdr}
%\usepackage{tgschola} % or any other font package you like

\pagestyle{fancy}
\fancyhf{}
\fancyhead[C]{%
  \footnotesize\sffamily
  \yourname\quad
  \textcolor{blue}{\itshape\yourweb}\quad
  \textcolor{black}{\youremail}}

\newcommand{\soptitle}{Personal Statement}
\newcommand{\yourname}{Ana Paula Centeno}
\newcommand{\youremail}{anapaula@cs.rutgers.edu}
\newcommand{\yourweb}{https://www.cs.rutgers.edu/\~{}anapaula}

\newcommand{\statement}[1]{\par\medskip
  \underline{\textcolor{blue}{\textbf{#1:}}}\space
}

\usepackage[
  colorlinks,
  breaklinks,
  pdftitle={\yourname - \soptitle},
  pdfauthor={\yourname},
  unicode
]{hyperref}

\begin{document}

\begin{center}\LARGE\soptitle\\
\large \yourname\ - January 2020
\end{center}

\hrule
\vspace{1pt}
\hrule height 1pt

\bigskip
\bigskip
When I started as a Part Time Lecturer in the fall of 2014 I never
dreamed that I would find such a fulfilling career as an
educator. Teaching Computer Architecture gave me the opportunity to
teach a subject that I enjoy and also the opportunity as a graduate
student to try a different career path other than research. I thought
research was exciting; what I didn't know was that teaching would give
me goose bumps. I find it extremely invigorating to partake knowledge
with students and to watch them learn with excitement about a science
field that I love.

\bigskip

Once a Teaching Professor I also engaged with students as an Academic
Advisor and SAS mentor, which gave me the inside look of the student's
life at Rutgers and some of the hardships students face. That changed
my approach to classroom teaching and assignments. Assignments are not
solely a way to solidify a student's understanding of a subject. A
carefully crafted assignment will motivate student learning and is
likely to improve major retention. That is specially relevant as the
Computer Science community looks to retain women in the major.

\bigskip

My role as the coordinator for the Introduction to Computer Science
course allowed me, together with my colleagues, to reorganize the
course. One of the improvements was the creating of a detailed set of
learning objectives that closely relate to the curriculum. By
providing students with learning objectives, they are able to assess
their own learning progress throughout the semester.

\bigskip

I am now part of the Advancing Women in Computer Science
initiative. DIMACS, the Department of Computer Science, and the
Douglass Residential College, partnered up to undertake an integrated
initiative aimed at increasing the percentage of undergraduate women
majoring and minoring in Computer Science at Rutgers and
elsewhere. The initiative has recently received gift from the Gates
Foundation to continue with the reorganization of the core courses in
the Computer Science major.

\bigskip

As I continue my journey as an educator at Rutgers I would like to
continue to have a positive impact on the Computer Science curriculum
and student's lives.

\end{document}


%How to Write an Effective Teaching Statement
%1. Identify your learning objectives
%What are the most important skills and habits of mind that you want students to learn in your classes? These might include: Developing students’ problem solving strategies
%Insuring that students understand foundational concepts
%Modeling expert problem solving
%Teaching students to work collaboratively Reduce math or science anxiety

%2. Explain—with concrete, specific examples—how you accomplish these goals.
%A teaching statement offers a chance to discuss the exciting, innovative, and effective things you do in the classroom. %Make sure you identify the most successful assignments and activities you have used in your classes. Explain:
%• How do you interest and engage students.
%• How do you help students understand difficult ideas and concepts.
%• How do you assess student learning.

%3. Identify, again with examples, challenges you have faced in the classroom and how you addressed them.
%You need to demonstrate that you have meaningful classroom experience and are well prepared for a full teaching load.

%4. Integrate strong, supportive statements from student course evaluations. 

%5. Explain how your research contributes to your teaching.

%6. Describe the courses you’d like to teach.

%7. Keep it short and succinct
%In a job application, a teaching statement should be no more than 1-2 pages long.
