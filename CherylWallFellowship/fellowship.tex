\documentclass[11pt]{article}
\usepackage[letterpaper,margin=1in]{geometry}
\usepackage{xcolor}
\usepackage{fancyhdr}
%\usepackage{tgschola} % or any other font package you like

\pagestyle{fancy}
\fancyhf{}
\fancyhead[C]{%
  \footnotesize\sffamily
  \yourname\quad
  \textcolor{blue}{\itshape\yourweb}\quad
  \textcolor{black}{\youremail}}

\newcommand{\soptitle}{Service Statement}
\newcommand{\yourname}{Ana Paula Centeno}
\newcommand{\youremail}{anapaula@cs.rutgers.edu}
\newcommand{\yourweb}{https://www.cs.rutgers.edu/\~{}anapaula}

\newcommand{\statement}[1]{\par\medskip
  \underline{\textcolor{blue}{\textbf{#1:}}}\space
}

\usepackage[
  colorlinks,
  breaklinks,
  pdftitle={\yourname - \soptitle},
  pdfauthor={\yourname},
  unicode
]{hyperref}

\begin{document}

\begin{center}\LARGE\soptitle\\
\large \yourname\ - March 2024
\end{center}

\hrule
\vspace{1pt}
\hrule height 1pt

\bigskip
\bigskip

During my tenure at Rutgers I have become more than a professor; I
became a role model that instills in students a sense of belonging in
the computing field and a mentor that fosters interest in computer
science.

Of the Rutgers computer science graduating student body, 18\% are
females. According to research, one of the top reasons female students
leave computer science is the perception that they do not belong in
the field. Therefore, it is of utmost importance to create an
environment in CS courses and student gathering places that is
welcoming to all students regardless of gender.

Incoming students have not been equally exposed to computing and
females tend to have even less computing background than their male
peers. In addition to nurturing a sense of belonging, it is equally
important to provide students with the academic support they need to
succeed.

For the past 5 years I have been working to (a) make the CS culture at
Rutgers more welcoming and supportive and (b) increase the quality and
quantity of academic support that the computer science major
introduction course sequence offers to our students. Students achieve
higher grades when they feel supported in their academic career. To
increase students' sense of support I have created additional course
content (over 100 learning objective videos) and a tutoring program
for the introductory sequence.

Following is a summary of my diversity, equity and inclusion
activities:


\bigskip

\begin{itemize}

  \item I have been one of the {\bf leaders in the Advancing Women in
    Computer Science (AWiCS) CS/DIMACS/Douglass Initiative that has
    brought in \$490K} from the Khoury Center for Inclusive Computing
    at Northeastern University.

  \item As one of AWiCS leaders at Rutgers, I am responsible for
    coordinating curricular revisions for the introduction course
    sequence and co-curricular programming.
    
  \item I have been instrumental in the complete redesign of the {\bf
    01:198:111 (Fall 2019)} and {\bf 01:198:112 (Spring 2021)}
    curricula, as part of an effort that has been directly supported
    by the Rutgers Chancellor's Office and Northeastern Grant.

  \item I am the course coordinator for 01:198:111 and 01:198:112,
    responsible for synchronized course content across lectures, the
    organization of 50 Learning Assistants and 40 graders. Current
    enrollment is 3600 students per Academic Year.
    
  \item I have been working with the CS department and SAS leadership
    to introduce a 3-course introduction to CS sequence to the major
    that will benefit students without computing background.
    
  \item I have created a {\bf High School outreach program} to promote
    diversity and inclusion. Students from minority serving high
    schools spend a day at the New Brunswick campus learning about the
    computer science major and what is to be a student at Rutgers.
        
  \item I created a mentorship program for freshman, Assignment Guru,
    that gives students the opportunity to gain skills while creating
    assignments that demonstrate the positive social impact of
    computing. Over 30 mentees per year. https://assignment-guru.cs.rutgers.edu
    
  \item I submitted NSF grants to support the CS education research that
    I have been conducting.
    
  \item My courses receive consistently high student
    ratings. Considering the reviews for both courses, my weighted
    mean for each question presented to the students is higher than
    the course's, department's, and level's weighted mean.
    
\end{itemize}

The previous activities are ongoing. On Spring 2024 I have
started mentoring four undergraduate students in conjunction with
Margaret Cozzens (DIMACS). This research group aims to have a deeper
understanding of the impact of our DEI efforts.

The course release will enable to continue to mentor students and lead
the creation of the 3-course introductory sequence for the department.

I have been recognized by my efforts to increase diversity, equity and
inclusion at Rutgers, winning two awards:

\begin{itemize}
  \item Presidential Employee Excellence Recognition Award - Rutgers Gateway Award for Service to Students, 2023. 
  \item Provost's Award for Excellence in STEM Diversity, 2021.
\end{itemize}



\end{document}
